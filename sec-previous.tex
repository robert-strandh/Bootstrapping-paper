\section{Previous work}

\subsection{Overview of existing techniques}

In his excellent paper describing how \sbcl{} is bootstrapped
\cite{Rhodes:2008:SSC:1482373.1482380}, Rhodes gives an overview of
how different existing \commonlisp{} systems are made to evolve.
Below, we summarize the contents of that paper.

We can roughly divide \commonlisp{} implementations into those that
are mostly written in some other language, and those that are mostly
written in \commonlisp{}.

In the first category, there are implementations that specifically
cater to applications written in that other language and that need
some scripting capabilities that are supplied by the \commonlisp{}
implementation.  Whether it is advantageous or not for these
implementations to be written mainly in that other language is outside
the scope of this paper.

Of the implementations in the second category that are currently
actively used, Rhodes claims%
\footnote{For the commercial \commonlisp{} implementations cited in
  the paper by Rhodes, he includes a disclaimer that only anecdotal
  evidence for this information is available.}
that \allegro{}, \lispworks{}, \cmucl{}, Scieneer, and \ccl{} are only
possible to build using older versions of the same system, and only
using image-based techniques.  Only \sbcl{} can be bootstrapped using
several other \commonlisp{} implementations.

Even a \commonlisp{} implementation that is largely written in
\commonlisp{} such as \sbcl{} has some amount of code written in other
languages.  In the case of \sbcl{}, Rhodes gives the number
$35\thinspace000$ lines of \clanguage{} and assembly code ``for
services such as signal handling and garbage collection'', of which
$8\thinspace000$ is for the garbage collector.  The remaining lines
can be summarized as around $2\thinspace000$ lines per operating
system supported.  This amount of code written in other languages is
very modest.

\subsection{Implementations written in some other language}

We are aware of the following widespread implementations written in a
language other than \commonlisp{}:

\begin{itemize}
\item CLISP is written in \clanguage{}, and uses the macro facility of
  \clanguage{} to create a domain-specific language that turns
  \clanguage{} into something slightly better adapted to writing a
  \commonlisp{} system.  The CLISP generates \emph{bytecodes} that are
  subsequently interpreted by code written in \clanguage{}.
\item \ecl{} uses a \clanguage{} compiler as its main backend.  The
  code generated from the \ecl{} compiler is \clanguage{} code which
  is then compiled to native code by that \clanguage{} compiler.  This
  technique makes it possible to generate reasonably fast code, while
  letting the \clanguage{} compiler be the only place where knowledge
  about the native machine code is encoded.  Furthermore, \ecl{} is
  designed to interoperate with \clanguage{} in that it uses the
  conventions of \clanguage{}, in particular the \clanguage{} calling
  conventions.  \ecl{} is also largely written in \clanguage{}.
\item \clasp{} is a relatively new implementation of \commonlisp{} and
  of somewhat unusual origin.  The main purpose of \clasp{} is to
  allow a nearly seamless interoperation with modules written in
  \cplusplus{}, mainly to allow the use of the CANDO system from
  \commonlisp{}.  This interoperation is possible due to the fact that
  LLVM is used as a backend, both for the \cplusplus{} code and for
  the \commonlisp{} code.  The author of \clasp{} cites this
  interoperation as the main reason why the system is written in
  \cplusplus{}.  The origin of the \cplusplus{} code is that the
  author manually translated the \clanguage{} code of the \ecl{}
  \commonlisp{} implementation to \cplusplus{}.
\end{itemize}

We have identified several different reasons why so many \commonlisp{}
implementations are written in some other language:

\begin{itemize}
\item At a time when no complete high-quality \commonlisp{} system
  existed, using a different language for the implementation was a
  necessity.  Several existing implementations originated at that
  time, and using \commonlisp{} instead of some other language would
  amount to a complete rewrite, the result of which would be a
  different system.
\item Determining the inner works of a new \commonlisp{} system
  requires decisions about low-level details of object
  representation.  Such representation requires a notation that can
  express these details.  In a \commonlisp{} system written entirely
  in \commonlisp{} these decisions would be part of the code generator
  of the compiler.  However, it seems more immediate to many people to
  encode these decisions more directly in a low-level language such as
  \clanguage{}.
\item Even though there is a long history of writing the language
  processor for a particular language in that very language, this
  possibility remains obscure to many implementers.
\end{itemize}
