\documentclass{sig-alternate-05-2015}
\usepackage[utf8]{inputenc}

\def\inputfig#1{\input #1}
\def\inputtex#1{\input #1}
\def\inputal#1{\input #1}
\def\inputcode#1{\input #1}

\inputtex{logos.tex}
\inputtex{refmacros.tex}
\inputtex{other-macros.tex}

\begin{document}
\setcopyright{rightsretained}
\title{Bootstrapping \commonlisp{}}
\numberofauthors{1}
\author{\alignauthor
Robert Strandh\\
\affaddr{University of Bordeaux}\\
\affaddr{351, Cours de la Libération}\\
\affaddr{Talence, France}\\
\email{robert.strandh@u-bordeaux1.fr}}


\maketitle

\begin{abstract}
The most common technique for building an executable \commonlisp{}
implementations, is to use a technique known as
\textit{bootstrapping}, i.e., the executable code is built from
\emph{source code} using a number of tools such a \emph{language
  processors}, shell tools, etc.  The \commonlisp{} implementations
that are not built this way evolve through careful modification of an
existing \emph{image}.

Most of the implementations that are built using bootstrapping have a
large component written in some lower-level language; typically
\clanguage{}.  The only widespread \commonlisp{} implementation
written mainly in \commonlisp{}, and that is built with bootstrapping,
is \sbcl{}

We argue that there is no particular reason why a \commonlisp{}
implementation should be written in anything other than \commonlisp{},
and we investigate reasons why so many existing implementations have
such a large fraction of the code written in languages that are not
particularly adapted for this task.
\end{abstract}

%% \begin{CCSXML}
%%   <ccs2012>
%%   <concept>
%%   <concept_id>10011007.10011006.10011008.10011024.10011027</concept_id>
%%   <concept_desc>Software and its engineering~Control structures</concept_desc>
%%   <concept_significance>500</concept_significance>
%%   </concept>
%%   </ccs2012>
%% \end{CCSXML}

%% \ccsdesc[500]{Software and its engineering~Control structures}

%% \printccsdesc

%\keywords{\clos{}, \commonlisp{}}

\inputtex{sec-introduction.tex}
\inputtex{sec-previous.tex}
%% \inputtex{sec-our-method.tex}
%% \inputtex{sec-benefits.tex}
%% \inputtex{sec-conclusions.tex}
%% \inputtex{sec-acknowledgments.tex}

\bibliographystyle{abbrv}
\bibliography{bootstrapping-common-lisp}
\end{document}
