\section{Introduction}
\label{sec-introduction}

For the purpose of this paper, by \emph{bootstrapping} a \commonlisp{}
system, we mean creating some \emph{target} \commonlisp{} system by
building it from its associated source code, using various
\emph{tools} and \emph{language processors} to transform that source
code into an \emph{executable file} for some typical operating system
such as GNU/Linux.  When a target \commonlisp{} system can be
bootstrapped this way, the typical way of making it evolve through
maintenance, is to modify its source code and then restart the
bootstrapping process to build an updated executable file.

Not all \commonlisp{} systems are created so that they can evolve this
way.  Some systems evolve by the careful modification of an existing
\emph{executing image} which is then saved as an executable file that
can be executed as usual.

Without arguing pros and cons of either technique, in this paper, we
will concentrate on systems built using bootstrapping.

Before we can start investigating different options for bootstrapping,
we must deal with an annoying but crucial detail, namely the
definition of \emph{source code}.  The Free Software Foundation
defines it as ``the preferred form of a work for making modifications
to it''.  We agree completely with this definition.  It excludes the
use of code that was automatically produced from code in some other
form.  In practice, it also excludes code written directly in machine
language and most code written in some assembly language, with the
exception of small code fragments that can not be expressed very
easily in some other language, and of code fragments that are part of
a code generator written in some higher-level language.

However, for it to be possible for the source code of a \commonlisp{}
system to be turned into an executable file, there must be
\emph{language processors} (i.e., compilers and/or interpreters)
available that can handle the languages used to create the source
code.  The main debate when it comes to bootstrapping techniques seems
to be what is meant by \emph{available} in this context.  A common
definition seems to be something like \emph{whatever is available on a
  GNU/Linux system out of the box}.

One of the consequences of such a definition of \emph{available} is
usually that, in order to write a \commonlisp{} system, one has to use
some programming language considered lower level than \commonlisp{}
itself.  Typically, \clanguage{} plays this role.

But choosing a language such as \clanguage{} for implementing
\commonlisp{} have many negative consequences.  The most immediate
consequence is of course that it is much harder to express the
semantics of \commonlisp{} in a language without the tools that we are
quite attached to, such as automatic memory management, first-class
functions, list processing, and (in particular) \clos{}.  Worse, even
though a large part of the system can be written in \commonlisp{},
because of the gradual bootstrapping process, many modules of the
final system must be written in a \emph{subset} of \commonlisp{}.  A
common such case occurs when the design of the final system is such
that \clos{} is added towards the end of the bootstrapping process,
thereby making it impossible to use one of the most powerful tools
available to the programmer for most of the system code.

In this paper, we argue that one of the main reasons of the creator(s)
for wanting a target \commonlisp{} system in the first place is
typically that they are convinced of the virtues of this language for
writing programs.  Furthermore, \commonlisp{} is uniquely well adapted
to writing language processors.  The obvious choice for a language for
writing a \commonlisp{} system is therefore \commonlisp{} itself.
Since there is now a multitude of good \commonlisp{} implementations
available and easily installable on most widely-used operating
systems, we think that \commonlisp{} should be considered to be a
language for which there are language processors available for
bootstrapping.
