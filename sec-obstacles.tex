\section{Obstacles}

There are several obstacles to bootstrapping a target \commonlisp{}
system on a host \commonlisp{} system.

\subsection{File compiler}

While writing an optimizing compiler for \commonlisp{} is no easy
task, the way a \emph{file compiler} operates is fairly simple, in
that it reads a file containing source code and writes a file
containing code with the equivalent semantics but in some other form,
typically native machine code.

Clearly, for reasons of maintenance, it is desirable for all compilers
to share the same source code, whether it be the native in-memory
compiler, the native file compiler, or the \emph{cross compiler},
i.e., the compiler running in the host \commonlisp{} system but
producing output for the target system.  In particular, all compilers
must access the \emph{global environment} for information such as
macro expanders, definitions of types, values of constants, etc.

The native compilers would typically access such environment
information using standard functions such as \texttt{macro-function},
\texttt{symbol-value}, etc.%
\footnote{Some such information is not available as standard
  functions.  For example, there is no standardized way of obtaining
  the expansion of a given type definition.}
However, this technique is not possible to use in the cross compiler,
because it would then obtain information about the host \commonlisp{}
system rather about the target \commonlisp{} system.

One way of avoiding this conundrum is to use packages with different
names for information about the target environment.  These packages
are ultimately renamed when the resulting code is incorporated into
the native target system.  The technique of package renaming is
heavily used by SBCL.

\subsection{Floating-point arithmetic at compile time}

Any good compiler contains code for doing \emph{constant folding}.
This optimization technique involves evaluating expressions in which
the arguments are known to be constant, and when the arguments are
floating-point numbers, it is important to make sure to use the
semantics defined by the target \commonlisp{} system of these
operations.  It is not possible to use the analogous operations in the
host because the \commonlisp{} standard allows for
implementation-specific variations such as the number of different
floating-point types, and the precision of those types.

If these operations are desirable, then the compiler must contain a
complete set of operations that emulate the desired target semantics.
However, it can be argued that such optimizations can be deferred
until the native system is able to compile its own code.
