\section{Proposed bootstrapping technique}
\label{sec-our-technique}

As with other systems such as \sbcl{}, we create a \emph{minimal
  system} that is then used to load additional object code in the form
of so-called \texttt{fasl} files generated by the compiler, running as
a cross compiler on the host system.

However, there is a major design consideration that needs to be taken
into account in order for this technique to work.  In order for it to
be possible to load object code that has the effect of defining or
redefining methods on generic functions, the native compiler needs to
be present in the system so that native code for new effective methods
and new discriminating functions can be generated.  But if the
compiler is written using generic functions, the minimal system must
contain the compiler, or else the compiler can not be loaded.

In \sbcl{}, this problem does not exist, because the compiler is
written so that it does not use generic functions.  Recall that
\sbcl{} is derived from \cmucl{} which was largely written before
\clos{} was part of the \commonlisp{} standard.  Thus, the compiler
was written without generic functions.

For a modern implementation of \commonlisp{}, generic functions
represent an essential tool for keeping the code modular and
maintainable.  However, it would still be desirable to have a minimal
system that does not contain the compiler.

The technique we propose is to generate the object code of the
compiler not as a compilation of individual source code files into
individual object code files, but to generate a single object code
file from the full compiler source.  When the full source code of the
compiler is present in the memory of the host system, we turn each
generic function, including all its methods, into an ordinary
(non-generic) function by combining the code from the individual
methods according to the rules of the method combination in question.
For this technique to work, we must know in advance what possible
combinations of classes will result in calls to some effective method,
and the exact composition of that effective method.  However, the code
for computing all these combinations is already present in the system
as a solution \cite{Strandh:2014:RMI:2635648.2635656} to the
metastability problem indicated by Kitzales et al
\cite{Kiczales:1991:AMP:574212}.

\cite{Strandh:2015:ELS:Environments}
